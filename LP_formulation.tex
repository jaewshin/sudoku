\documentclass[12pt]{article}
\usepackage{graphicx}
\usepackage{fullpage}
\usepackage[normalem]{ulem}
\usepackage{fancyhdr,graphicx,amsmath,amssymb, mathtools, scrextend, titlesec, enumitem}




\title{IP formulation of Sudoku Puzzle}
\begin{document}

	\section{Assumption}
	This Integer Programming formulation is designed specifically for Sudoku puzzle of size 9*9. However, theoretically any Sudoku puzzle of size n by n is solvable with this model. 
	
	\section{IP formulation}
	Suppose that i represents the order of rows and j represents the order of columns. k is a possible value of an entry in the puzzle that is in the range from 1 to 9. The formulation is as follows: 
	\linebreak\linebreak
	\centering{$x_{ijk} = 
				\begin{cases}
					1 & \text{if ith row, jth column element is k}\\
					0 & \text{otherwise} 
				\end{cases}$
	\linebreak\linebreak
	\centering{G = set of all known entries of the puzzle}
	\linebreak\linebreak
	\centering{$\sum_{i} x_{ijk} = 1$ for $j = 1, ... , 9$, for $k = 1, ..., 9$}
	\linebreak\linebreak
	\centering{$\sum_{j} x_{ijk} = 1$ for $i = 1, ... , 9$, for $k = 1, ..., 9$}
	\linebreak\linebreak
	\centering{$\sum_{k} x_{ijk} = 1$ for $i = 1, ..., 9$, for $j = 1, ... , 9$}
	\linebreak\linebreak
	\centering{$\sum_{p}^{p+2} \sum_{q}^{q+2} x_{pqk} = 1$, for $p = 3i-2$, $q = 3j-2$, for $i, j = 1, 2, 3$}
	

\end{document}